\chapter{HASIL DAN PEMBAHASAN}

\section{HALAMAN AUTENTIKASI}
Berisi halaman yang menampilkan dua opsi untuk login. Login manual maupun menggunakan Google OAuth untuk mempermudah proses autentikasi.

\begin{figure}[htbp]
    \centering
    \includegraphics[width=0.8\textwidth]{images/halaman_login.png}
    \caption{Halaman Login}
    \label{fig:halaman_login}
\end{figure}

\section{HALAMAN DONASI}
\subsection{Form Donasi}
Halaman donasi menyediakan input seperti nama donor, nominal donasi, pesan serta opsi media share. Sistem melakukan validasi dasar sebelum melanjutkan proses ke Midtrans.

\begin{figure}[htbp]
    \centering
    \includegraphics[width=0.8\textwidth]{images/form_donasi.png}
    \caption{Form Donasi}
    \label{fig:form_donasi}
\end{figure}

\subsection{Halaman Pembayaran Midtrans}
Setelah form donasi divalidasi, sistem mengirimkan request Snap Token ke Midtrans. Donor kemudian diarahkan ke halaman pembayaran yang disediakan Midtrans.

\begin{figure}[htbp]
    \centering
    \includegraphics[width=0.8\textwidth]{images/pembayaran_midtrans.png}
    \caption{Halaman Pembayaran Midtrans}
    \label{fig:pembayaran_midtrans}
\end{figure}

\section{DASHBOARD KREATOR}
Pada dashboard kreator menampilkan statistik utama seperti total donasi, total pendapatan, menu riwayat donasi, dan menu leaderboard donasi terbanyak perbulan.

\begin{figure}[htbp]
    \centering
    \includegraphics[width=0.8\textwidth]{images/dashboard_kreator.png}
    \caption{Dashboard Kreator}
    \label{fig:dashboard_kreator}
\end{figure}

\section{HALAMAN REQUEST PAYOUT}
Kreator dapat mengajukan pencairan dana berdasarkan saldo yang tersedia. Sistem menghitung saldo bersih (total donasi PAID yang belum pernah dipayout).

\begin{figure}[htbp]
    \centering
    \includegraphics[width=0.8\textwidth]{images/halaman_payout.png}
    \caption{Halaman Payout}
    \label{fig:halaman_payout}
\end{figure}

\section{OVERLAY}
Overlay berfungsi sebagai antarmuka yang ditampilkan pada platform streaming melalui OBS. Seluruh elemen overlay diambil secara \textit{real-time} dari API sehingga kreator dapat menampilkan interaksi donasi secara langsung kepada penonton. Sub bab ini menampilkan implementasi setiap komponen overlay.

\subsection{Overlay Notifikasi Donasi}
Overlay menampilkan notifikasi donasi secara \textit{real-time} yang diambil melalui polling API tampilan ini dikonfigurasi agar kompatibel dengan OBS untuk keperluan streaming.

\begin{figure}[htbp]
    \centering
    \includegraphics[width=0.6\textwidth]{images/notifikasi_donasi.png}
    \caption{Notifikasi Donasi}
    \label{fig:notifikasi_donasi}
\end{figure}

\subsection{Overlay Media Share}
Overlay Media Share menampilkan video YouTube yang diputar berdasarkan permintaan donor. Pada tampilan ini, video akan muncul di area utama layar, sementara informasi donasi – seperti nama donor, nominal, dan pesan singkat – tampil di bagian bawah kiri.

\begin{figure}[htbp]
    \centering
    \includegraphics[width=0.6\textwidth]{images/media_share.png}
    \caption{Media Share}
    \label{fig:media_share}
\end{figure}

\subsection{Overlay QR link Donasi}
Overlay menyediakan kode QR statis yang mengarahkan penonton langsung ke halaman donasi kreator. QR ini memfasilitasi penonton streamer untuk berdonasi tanpa harus mengetik tautan secara manual.

\begin{figure}[htbp]
    \centering
    \includegraphics[width=0.4\textwidth]{images/qr_donasi.png}
    \caption{QR}
    \label{fig:qr_donasi}
\end{figure}

\subsection{Overlay Leaderboard}
Overlay ini menampilkan leaderboard daftar pendukung terbesar (\textit{top donors}) atau ranking total donasi yang telah dioptimasi dengan limit dan sorting. Tampilan ini digunakan untuk memberikan apresiasi kepada donatur selama streaming.

\begin{figure}[htbp]
    \centering
    \includegraphics[width=0.6\textwidth]{images/leaderboard_donatur.png}
    \caption{Leaderboard Donatur}
    \label{fig:leaderboard_donatur}
\end{figure}

\section{DASHBOARD ADMIN}
Bagian ini menampilkan antarmuka yang digunakan admin untuk mengelola aktivitas pada platform, mulai dari pemantauan data donasi hingga pengelolaan creator dan proses payout. Dashboard admin terdiri dari tiga menu utama: Dashboard, Creator, dan Payout.

\subsection{Halaman Dashboard}
Halaman dashboard menampilkan ringkasan statistik utama sistem, seperti jumlah creator terdaftar, jumlah pengajuan payout, serta jumlah payout yang telah diselesaikan. Selain itu, halaman ini juga memvisualisasikan Top 5 Creator Paling Aktif berdasarkan total donasi yang diterima.

\begin{figure}[htbp]
    \centering
    \includegraphics[width=0.8\textwidth]{images/dashboard_admin.png}
    \caption{Dashboard Admin}
    \label{fig:dashboard_admin}
\end{figure}

\subsection{Halaman Creator}
Halaman creator digunakan untuk melihat daftar creator yang terdaftar pada platform, termasuk informasi status kelengkapan data payout masing-masing. Admin dapat melakukan pencarian creator dan melihat detail lengkap akun creator melalui dialog Detail Creator.

\begin{figure}[htbp]
    \centering
    \includegraphics[width=0.8\textwidth]{images/tabel_creator.png}
    \caption{Tabel Daftar Creator dan Detail Creator}
    \label{fig:tabel_creator}
\end{figure}

\subsection{Halaman Payout}
Halaman payout menampilkan daftar pengajuan pencairan dana yang dilakukan oleh para creator. Admin dapat memantau nominal, waktu pengajuan, status proses, serta catatan yang terkait dengan setiap payout. Halaman ini membantu admin memastikan bahwa seluruh proses pencairan berjalan dengan transparan dan terdokumentasi.

\begin{figure}[htbp]
    \centering
    \includegraphics[width=0.8\textwidth]{images/tampilan_payout.png}
    \caption{Tampilan Payout}
    \label{fig:tampilan_payout}
\end{figure}

\section{STRUKTUR BASIS DATA}
Sistem menggunakan MongoDB sebagai basis data utama. Setiap fitur ini menghasilkan koleksi tersendiri untuk memudahkan proses penyimpanan, pemantauan, dan analisis data. Koleksi yang terbentuk dan fungsinya dapat dilihat pada Tabel \ref{tab:struktur_database}.

\begin{table}[htbp]
    \centering
    \caption{Struktur Basis Data Nyumbangin}
    \label{tab:struktur_database}
    \begin{tabularx}{\textwidth}{|l|X|}
        \hline
        \textbf{Koleksi} & \textbf{Deskripsi Fungsi} \\ \hline
        admins & Menyimpan data akun admin untuk panel pengelolaan. \\ \hline
        contacts & Menyimpan feedback atau saran/keluhan dari pengguna. \\ \hline
        creators & Menyimpan informasi kreator, data payout, dan profil. \\ \hline
        donations & Mencatat transaksi sementara sebelum divalidasi. \\ \hline
        donationhistories & Menyimpan riwayat donasi yang sudah diproses (PAID). \\ \hline
        donationshares & Mencatat data pembagian link donasi. \\ \hline
        filteredwords & Menyimpan daftar kata-kata terlarang (sensor). \\ \hline
        mediashares & Menyimpan request video YouTube untuk diputar. \\ \hline
        monthlyleaderboards & Menyimpan agregasi peringkat donatur bulanan. \\ \hline
        notifications & Menyimpan data notifikasi real-time untuk overlay. \\ \hline
        payouts & Mencatat pengajuan dan proses pencairan dana. \\ \hline
        profileimages & Menyimpan metadata gambar profil kreator. \\ \hline
    \end{tabularx}
\end{table}
