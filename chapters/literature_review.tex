\chapter{LANDASAN TEORI}

\section{PLATFORM DONASI DIGITAL}
Platform donasi digital merupakan pengembangan dari teknologi platform berbasis internet yang memfasilitasi interaksi antara berbagai pihak untuk tujuan sosial dan filantropi. Platform digital didefinisikan sebagai seperangkat komponen teknologi yang menyediakan fungsi inti bagi suatu sistem dan menjadi fondasi bagi pengembangan layanan pelengkap di atasnya (Shneor et al., n.d.; Zhou \& Inoue, 2025). Secara konseptual, platform ini beroperasi sebagai \textit{two-sided market} yang mempertemukan kelompok pengguna berbeda namun saling bergantung, seperti donatur dan penerima manfaat, di mana nilai platform tercipta dari interaksi antar pengguna tersebut (Jullien et al., 2021).

Dalam konteks filantropi di Indonesia, platform digital digunakan sebagai alternatif lembaga amil konvensional dengan menawarkan kemudahan akses, transparansi, dan kecepatan distribusi dana untuk zakat, infaq, sedekah, dan wakaf bagi para donator (Febriandika et al., 2024; Hidayatullah \& Purbasari, 2022). Perkembangan ini sejalan dengan meningkatnya kepercayaan masyarakat terhadap sistem donasi daring serta pergeseran perilaku filantropi ke media digital (Indriyani \& Ibrahim, 2024).

Dalam lingkup penggalangan dana massal, platform \textit{donation-based crowdfunding} didefinisikan secara spesifik sebagai perantara teknologi yang digunakan oleh penggalang dana (\textit{fundraisers}) untuk mencocokkan donasi dengan tujuan para donatur. Tujuan yang dipertemukan dalam platform ini tidak bersifat moneter, melainkan pemenuhan kebutuhan psikologis donatur (Fadzirul Kamarubahrin et al., n.d.). Dalam ekosistem ini, penyedia situs web (\textit{website providers}) berperan menyediakan layanan web dan membangun sistem yang mendukung pemilik proyek (\textit{founders}) untuk mempresentasikan kampanye mereka kepada calon pendukung (\textit{backers}) guna menggalang donasi (Fadzirul Kamarubahrin et al., n.d.; Sirisawat et al., 2022).

\section{ARSITEKTUR APLIKASI WEB MODERN}
Bagian ini membahas konsep dasar arsitektur aplikasi web modern yang menjadi landasan dalam memahami cara kerja sistem berbasis web (Fraihat et al., 2022). Pemahaman mengenai pola komunikasi serta pembagian lapisan dalam aplikasi diperlukan untuk menjelaskan bagaimana komponen penyusun sistem saling berinteraksi dan menjalankan fungsinya. Oleh karena itu, pembahasan berikut difokuskan pada model \textit{client-server} dan arsitektur \textit{three-tier/N-tier} sebagai struktur arsitektural yang umum digunakan dalam pengembangan aplikasi web masa kini.

\subsection{Arsitektur Three-Tier/N-Tier}
Arsitektur \textit{three-tier} (tiga lapisan) atau N-tier merupakan sebuah model arsitektur perangkat lunak yang membagi fungsionalitas aplikasi menjadi tiga lapisan logis dan fisik yang berbeda untuk meningkatkan skalabilitas dan keandalan sistem (Maruf \& Ugli, n.d.; Prabu \& De Paul, 2025). Tiga lapisan utama tersebut terdiri dari: lapisan presentasi (\textit{presentation tier}), yang berinteraksi langsung dengan pengguna; lapisan aplikasi (\textit{application tier} atau \textit{business logic tier}), yang menangani pemrosesan data dan logika bisnis inti; dan lapisan data (\textit{data tier}), yang bertanggung jawab atas penyimpanan dan manajemen basis data (Maruf \& Ugli, n.d.). Pemisahan fungsionalitas ini memungkinkan setiap lapisan dikelola dan dikembangkan secara independen, menjadi arsitektur ini pilihan yang efektif untuk sistem yang memerlukan ketersediaan tinggi (\textit{high availability}), seperti pada kasus penerapan LMS Moodle (Ismail, 2023).

\subsection{Konsep Client-Server}
\textit{Client-Server} adalah sebuah model perangkat lunak yang memungkinkan sumber daya dan permintaan layanan dipenuhi melalui jaringan, dimana komputer yang disebut klien (\textit{client}) akan meminta sumber daya atau layanan, dan server akan menerima permintaan tersebut, memprosesnya, dan memberikan respons yang sesuai (Assistant professor, n.d.; Geofrey et al., n.d.). Model ini memungkinkan banyak pengguna untuk secara simultan mengakses dan menggunakan sumber daya yang disimpan secara terpusat di server, yang biasanya memiliki basis data dan menjalankan program untuk memproses permintaan (Geofrey et al., n.d.). Komunikasi antara klien dan server difasilitasi melalui protokol standar seperti HTTP, FTP, dan SMTP, dan model ini memberikan inter-proses komunikasi yang memungkinkan pertukaran data, menjadikannya fondasi bagi banyak aplikasi termasuk email, sistem basis data, dan internet (Nyabuto, 2023; Assistant professor, n.d.).

\section{REST API DAN PROTOKOL HTTP}
REST API dipahami sebagai pendekatan arsitektur \textit{web service} yang memanfaatkan prinsip \textit{Representational State Transfer} (REST) (Ehsan et al., 2022; Roziqin et al., 2023). REST menekankan penggunaan URI standar untuk mengidentifikasi \textit{resource}, memanfaatkan protokol dan prinsip yang sudah ada di web, serta menerapkan batasan seperti \textit{addressability}, \textit{statelessness}, \textit{uniform interface}, dan \textit{representations} untuk memastikan interaksi yang sederhana namun kuat dalam sistem terdistribusi (Ehsan et al., 2022). Dalam konteks ini, HTTP berperan sebagai protokol utama yang digunakan baik sebagai standar komunikasi maupun sebagai media transportasi data (Roziqin et al., 2023). HTTP memungkinkan klien dan server bertukar informasi melalui metode seperti GET, POST, PUT, dan DELETE, dan menjadi fondasi bagi layanan REST karena sifatnya yang terbuka, sederhana, serta telah lama menjadi protokol inti web modern (Apriyani \& Hamdana, n.d.; Roziqin et al., 2023).

\section{AUTENTIKASI DAN OTORISASI}
Bagian ini membahas konsep dasar autentikasi dan otorisasi yang menjadi fondasi penting dalam pengamanan aplikasi berbasis web. Mekanisme pengenalan identitas pengguna dan pemberian hak akses harus dirancang secara tepat agar interaksi antar sistem tetap aman, terukur, dan sesuai dengan tingkat kewenangan yang dibutuhkan. Oleh karena itu, pembahasan berikut mencakup OAuth 2.0 sebagai protokol delegasi akses, OpenID Connect sebagai lapisan identitas, JSON Web Token (JWT) sebagai format token yang umum digunakan, skema Bearer Token yang banyak diadopsi dalam komunikasi API, serta prinsip-prinsip keamanan API yang memastikan perlindungan terhadap ancaman dan penyalahgunaan akses.

\subsection{OAuth 2.0}
OAuth 2.0 didefinisikan sebagai \textit{framework} otorisasi yang populer yang memungkinkan suatu aplikasi memperoleh akses terbatas ke \textit{resource} yang dilindungi tanpa harus mengetahui atau menyimpan kredensial pengguna secara langsung (Lodderstedt et al., 2025). OAuth 2.0 menyediakan seperangkat \textit{authorization server} (Philippaerts et al., 2022). \textit{Framework} ini dirancang untuk mendukung berbagai konteks---mulai dari aplikasi web, \textit{single-page apps}, hingga aplikasi mobile---dengan cara memberikan fleksibilitas pada mekanisme autentikasi dan otorisasi yang aman di antara berbagai jenis klien (Singh \& Chaudhary, 2023).

\subsection{OpenID Connect}
OpenID Connect (OIDC) merupakan sebuah protokol yang mapan yang digunakan secara luas dalam manajemen identitas terfederasi (\textit{federated identity management}). Protokol ini berfungsi sebagai dasar bagi otentikasi dan sistem Masuk Tunggal (\textit{Single Sign-On} atau SSO), yang memungkinkan klien untuk memverifikasi identitas pengguna akhir berdasarkan otentikasi yang dilakukan oleh Penyedia Identitas (\textit{Identity Provider}) (Hammann et al., 2020; Yasuda et al., 2022). Dibangun di atas kerangka kerja otorisasi OAuth 2.0, kegunaan OIDC meluas hingga ke infrastruktur kompleks, seperti memfasilitasi akses Secure Shell (SSH) pada pengaturan terfederasi dengan menggunakan token akses OIDC untuk otentikasi pengguna pada server jarak jauh (Gudu et al., 2025).

\subsection{JWT}
JSON Web Token (JWT) merupakan sebuah standar terbuka yang didasarkan pada RFC 7519, yang digunakan secara luas sebagai mekanisme standar untuk otentikasi dan otorisasi pengguna pada layanan web. Standar ini tidak hanya populer untuk mengamankan transmisi data dan otentikasi pada RESTful API, tetapi juga dapat diperluas untuk meningkatkan keamanan dengan menyimpan informasi historis perilaku pengguna, seperti konsistensi alamat IP dan jenis \textit{user agent} (Bucko et al., 2023; Rahman et al., n.d.). Sementara itu, JWT secara fundamental adalah format token yang memfasilitasi transmisi data yang ringkas dan aman antara pihak-pihak yang berkepentingan sebagai objek JSON, yang menjadikannya mekanisme otentikasi yang penting dalam implementasi berbagai aplikasi modern (Nashikhuddin et al., 2023).

\subsection{Skema Bearer Token}
Skema Bearer Token merupakan mekanisme autentikasi pada OAuth 2.0 di mana klien cukup menyertakan token pada header (\texttt{Authorization: Bearer <token>}) untuk memperoleh akses ke \textit{resource} yang dilindungi (Lodderstedt et al., 2025). Karena token ini bersifat \textit{bearer}, siapa pun yang memilikinya dapat menggunakannya tanpa verifikasi tambahan, sehingga membuat keamanan transport menjadi aspek kritis. Penelitian terbaru menyoroti bahwa risiko pencurian token dapat diminimalkan melalui penggunaan kanal terenkripsi, pembatasan masa hidup token, serta validasi ketat pada sisi server (Ball et al., n.d.). Selain itu, praktik modern juga menekankan pentingnya menghindari pengiriman token melalui URL dan memastikan proses otorisasi mengikuti pedoman keamanan OAuth 2.0 (Neelan, 2022).

\subsection{Keamanan API}
Keamanan API merupakan aspek kritis dalam pengembangan aplikasi modern karena API menjadi jalur utama pertukaran data dan sering menjadi target serangan (Chandramouli \& Butcher, 2020). Banyak celah keamanan muncul akibat pengelolaan aset API yang lemah, API lama yang tidak terinventarisasi, serta kerentanan pada alur data dan logika bisnis (Sun et al., 2022). Selain itu, meningkatnya kompleksitas arsitektur RESTful dan GraphQL memperluas permukaan serangan, termasuk risiko seperti \textit{information leakage}, \textit{unauthorized access}, dan eksploitasi validasi input yang tidak memadai (Zhao, n.d.).

Untuk mengatasi ancaman tersebut, mekanisme keamanan API membutuhkan pendekatan berlapis yang mencakup autentikasi kuat berbasis OAuth/JWT, penggunaan HTTPS/TLS untuk mengamankan transmisi data, serta manajemen hak akses yang detail guna mencegah penyalahgunaan kredensial (Zhao, n.d.). Pentingnya teknik seperti \textit{asset discovery}, \textit{traffic auditing}, dan analisis alur data untuk mengidentifikasi API tersembunyi dan aktivitas mencurigakan (Sun et al., 2022). Di samping itu, penggunaan API Gateway dapat membantu menerapkan pembatasan trafik, filtrasi permintaan, dan perlindungan terhadap serangan seperti DDoS, sehingga API tetap terawasi dan terlindungi secara konsisten.

\section{DATABASE NoSQL (MongoDB)}
Basis data dokumen NoSQL (\textit{Not Only SQL}) muncul sebagai alternatif yang signifikan terhadap basis data relasional tradisional yang sering memiliki batasan ketat pada struktur data dan relasi, sehingga kurang efisien untuk menangani volume data yang sangat besar (\textit{huge database}) (Byali et al., 2022). NoSQL document database mengatasi masalah ini dengan menyediakan kemampuan untuk menyimpan dan mengelola data dalam format dokumen, sehingga dapat menampung data yang tidak terstruktur, semi-struktur, maupun terstruktur (Carvalho et al., 2023). Keunggulan utama NoSQL, khususnya jenis berorientasi dokumen seperti MongoDB, terletak pada fleksibilitas dan skalabilitas horizontal yang tinggi, menjadikannya pilihan esensial ketika skema data yang dinamis tidak sesuai dengan kebutuhan basis data relasional (Byali et al., 2022).

\section{FLOWCHART SISTEM}
Flowchart merupakan salah satu model yang paling mendasar dan penting dalam perancangan sistem informasi, di mana ia berfungsi untuk mendesain dan menyusun alur dokumen serta memvisualisasikan prosedur atau tahapan proses secara sistematis. Secara umum, flowchart memiliki aplikasi yang luas di berbagai bidang seperti pengembangan perangkat lunak, desain teknik, dan eksperimen ilmiah (Zhang et al., 2023). Struktur data flowchart tradisional sering kali didasarkan pada \textit{adjacency list}, \textit{cross-linked list}, atau \textit{adjacency matrix} dari struktur graf, yang didasari fakta bahwa setiap dua node dapat memiliki hubungan koneksi (Zhang et al., 2023). Namun, terlepas dari kompleksitas penyimpanan datanya, flowchart tetap menjadi alat fundamental yang menyediakan representasi visual dari urutan dan hubungan logis dalam suatu sistem (Ratumurun et al., n.d.; Pan et al., 2024).

\section{UNIFIED MODELING LANGUAGE (UML)}
Unified Modeling Language (UML) didefinisikan sebagai sebuah bahasa pemodelan standar yang digunakan untuk merancang dan mendokumentasikan sistem berorientasi objek. Sebagai bahasa standar, UML menyediakan seperangkat notasi grafis yang komprehensif untuk memvisualisasikan, menspesifikasikan, membangun, dan mendokumentasikan artefak dalam sistem perangkat. Tujuan utama penggunaan UML adalah untuk memperjelas model yang tidak konsisten dan mengurangi ambiguitas selama proses pengembangan perangkat lunak (Amani Bestari et al., 2024). UML membantu memvisualkan, menspesifikasikan dan mendokumentasikan desain sistem secara grafis (Siska Narulita et al., 2024).

Dengan menggunakan diagram-diagram yang berbeda, seperti Use Case Diagram dan Activity Diagram, UML membantu pengembang dalam memodelkan interaksi, struktur, dan perilaku sistem (Marwah M. A. Dabdawb, 2024). Penerapan UML sangat krusial dalam siklus hidup pengembangan sistem (\textit{System Development Life Cycle} atau SDLC) karena membantu memastikan konsistensi model dan mempermudah komunikasi antara pihak-pihak yang terlibat dalam proyek (Marchezan et al., 2023).

\subsection{Use Case Diagram}
Use Case adalah suatu diagram fundamental yang umum diajarkan dalam ilmu komputer dan rekayasa perangkat lunak. Diagram ini berfungsi sebagai representasi visual dari fungsionalitas sistem dari sudut pandang pengguna. Meskipun definisinya tampak sederhana, penilaian terhadap diagram use case sering kali menjadi hambatan dalam proses pembelajaran, terutama karena dua masalah utama: masalah interpersonal (tidak adanya standar penilaian di antara para pengajar) (Jebli et al., 2024) dan masalah intrapersonal (inkonsistensi seorang pengajar saat menilai banyak diagram) (Fauzan et al., 2021; Abbott et al., 2025; Wang et al., 2025).

\subsection{Activity Diagram}
Activity Diagram adalah salah satu diagram perilaku yang tersedia dalam Unified Model Language (UML) yang digunakan untuk memodelkan alur kontrol dan alur data dalam suatu sistem (Sandfreni et al., 2021). Diagram ini secara visual merepresentasikan langkah-langkah, keputusan, dan urutan tindakan yang diperlukan untuk menyelesaikan suatu proses atau kegiatan bisnis tertentu (Siska Narulita et al., 2024). Dalam konteks pemodelan sistem, Activity Diagram sangat berguna untuk memvisualisasikan bagaimana berbagai kegiatan saling terkait dan bergantung satu sama lain (Jha et al., 2023; Ramdany et al., n.d.).

\subsection{Sequence Diagram}
Sequence Diagram adalah diagram UML yang paling umum kedua, digunakan untuk merepresentasikan interaksi objek dan pertukaran pesan antar objek tersebut seiring berjalannya waktu (Siska Narulita et al., 2024). Diagram ini secara visual menunjukkan bagaimana peristiwa atau aktivitas yang ada dalam sebuah use case dipetakan menjadi operasi-operasi dari kelas objek yang ada pada Class Diagram (Al-Fedaghi, n.d.).

\subsection{Class Diagram}
Class Diagram merupakan salah satu diagram Unified Modeling Language (UML) yang paling umum digunakan dalam pendidikan dan pengembangan perangkat lunak berorientasi objek (Siska Narulita et al., 2024b). Fungsi utama dari Class Diagram adalah untuk merepresentasikan kelas-kelas dalam sistem perangkat lunak dan hubungan yang terjalin antar kelas-kelas tersebut (Fauzan et al., 2021b).

\section{ENTITY-RELATIONSHIP DIAGRAM (ERD)}
Entity-Relationship Diagram (ERD) adalah salah satu teknik utama yang digunakan dalam perancangan basis data dan merupakan representasikan diagramatik utama dari model data konseptual (Pulungan et al., n.d.). Fungsi utamanya adalah untuk merefleksikan persyaratan data pengguna dalam suatu sistem basis data (PENGANTAR BASIS DATA, n.d.). ERD adalah tahap pertama dalam proses desain basis data dan memvisualisasikan bagaimana berbagai komponen data diatur dan berinteraksi (Jaimez-González \& Martínez-Samora, 2020). Dalam membuat ERD, beberapa hal penting harus dipertimbangkan, antara lain setiap basis data harus memiliki entitas (\textit{entities}) yang saling terhubung melalui hubungan (\textit{relationship}), dan setiap entitas harus memiliki atribut (\textit{attributes}) yang terdiri dari kunci utama (\textit{primary key}) dan kunci asing (\textit{foreign key}) (Afiifah et al., 2022).

\section{PERAN FITUR SOCIAL SHARING DALAM DONASI DIGITAL}
Kepercayaan merupakan faktor kunci dalam keberhasilan platform donasi digital, mengingat interaksi antara donatur dan penerima dilakukan sepenuhnya secara daring tanpa kontak langsung (Tarigan, 2023). Tingkat kepercayaan ini sangat dipengaruhi oleh persepsi transparansi dan kredibilitas sistem yang disediakan oleh platform donasi digital (Greselda Gosal et al., n.d.). Oleh karena itu, platform donasi perlu menyediakan mekanisme yang mampu memperkuat transparansi tersebut.

Salah satu mekanisme yang dapat diterapkan adalah fitur \textit{social sharing}, yang memungkinkan aktivitas donasi dibagikan ke media sosial. Penerapan fitur \textit{social sharing} berperan sebagai \textit{social proof}, di mana visibilitas partisipasi pengguna lain dapat meningkatkan persepsi keandalan dan legitimasi platform (Rahmayanti et al., 2024). Peningkatan kepercayaan dan persepsi positif tersebut pada akhirnya berdampak pada meningkatnya niat dan partisipasi pengguna dalam kegiatan donasi digital (Kamarudin et al., 2023).

Berdasarkan temuan tersebut, fitur \textit{social sharing} pada aplikasi Nyumbangin dirancang sebagai fitur pendukung yang memungkinkan pengguna membagikan aktivitas donasi ke media sosial. Fitur ini diharapkan dapat memperkuat kepercayaan pengguna serta mendorong partisipasi donasi secara lebih luas.
