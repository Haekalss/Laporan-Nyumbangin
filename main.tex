\documentclass[12pt, a4paper]{report}

% Required Packages
\usepackage[utf8]{inputenc}
\usepackage[T1]{fontenc}
\usepackage{mathptmx}   % Times New Roman for text
\usepackage{courier}    % Courier for code
\usepackage{helvet}     % Helvetica for sans-serif
\usepackage{amsmath}
\usepackage[english]{babel}
% Manual localization
\addto\captionsenglish{
  \renewcommand{\contentsname}{DAFTAR ISI}
  \renewcommand{\listfigurename}{DAFTAR GAMBAR}
  \renewcommand{\listtablename}{DAFTAR TABEL}
  \renewcommand{\bibname}{DAFTAR PUSTAKA}
  \renewcommand{\chaptername}{BAB}
  \renewcommand{\figurename}{Gambar}
  \renewcommand{\tablename}{Tabel}
}
\usepackage{geometry}
\usepackage{setspace}
\usepackage{titlesec}
\usepackage{graphicx}
\usepackage{hyperref}
\usepackage{booktabs}
\usepackage{array}
\usepackage{caption}
\usepackage{enumitem}
\usepackage{parskip}
\usepackage{fancyhdr}
\usepackage{tocloft}
\usepackage{tabularx}
\usepackage{longtable}
\usepackage{ragged2e}
\usepackage[backend=bibtex,style=numeric]{biblatex}
\usepackage{xstring}

% Page Geometry
\geometry{
    left=4cm,
    top=3cm,
    right=3cm,
    bottom=3cm
}

% Spacing
\onehalfspacing

% Bibliography Resource
\addbibresource{references.bib}

% Custom Commands
\newcommand{\citen}[1]{\cite{#1}}

\begin{document}

% --- FRONT MATTER ---
\begin{titlepage}
    \centering
    \vspace*{1cm}
    {\Large \textbf{NYUMBANGIN: PLATFORM DONASI DIGITAL}}\\[1cm]
    {\large \textbf{LAPORAN PROYEK III}}\\[1.5cm]
    Diajukan untuk Memenuhi Kelulusan Matakuliah\\
    Proyek 3 pada Program Studi DIV Teknik Informatika\\[2cm]
    
    \textbf{DISUSUN OLEH :}\\[0.5cm]
    \begin{tabular}{ll}
    714230027 & Muhamad Haekal Syukur \\
    714230060 & Muhammad Ferdy Leoza
    \end{tabular}\\[2cm]
    
    \textbf{PROGRAM STUDI DIV TEKNIK INFORMATIKA}\\
    \textbf{FAKULTAS SEKOLAH VOKASI}\\
    \textbf{UNIVERSITAS LOGISTIK \& BISNIS INTERNASIONAL}\\
    \textbf{BANDUNG}\\
    \textbf{2026}
\end{titlepage}

\chapter*{LEMBAR PENGESAHAN}
\begin{center}
    \textbf{NYUMBANGIN: PLATFORM DONASI DIGITAL}
\end{center}
\vspace{1cm}
Laporan Proyek 3 ini telah diperiksa, disetujui, dan disidangkan\\
Di Bandung,\\[1cm]

\noindent
\begin{tabular}{p{7cm} p{7cm}}
    Oleh: & \\
    Penguji Pendamping, & Penguji Utama, \\[2.5cm]
    \textbf{Rolly Maulana Awangga, S.T.,MT.} & \textbf{Rolly Maulana Awangga, S.T.,MT.} (Simulasi) \\
    NIK: 113.74.163 & NIK: 1 \\
    & \\
    Pembimbing, & Koordinator Proyek 3 \\[2.5cm]
    \textbf{Rolly Maulana Awangga, S.T.,MT.} & \textbf{Roni Habibi, S.Kom., M.T., SFPC} \\
    NIK: 117.86.219 & NIK: 117.88.233 \\
\end{tabular}

\vspace{1cm}
\begin{center}
    Menyetujui,\\
    Ketua Program Studi D-IV Teknik Informatika,\\[2.5cm]
    \textbf{Roni Andarsyah, S.T., M.Kom}\\
    NIK: 115.88.193
\end{center}

\chapter*{SURAT PERNYATAAN BEBAS PLAGIARISME}
\begin{tabular}{ll}
Nama & : Muhamad Haekal Syukur \\
NPM & : 714230027 \\
Program Studi & : DIV Teknik Informatika \\
Judul & : NYUMBANGIN: PLATFORM DONASI DIGITAL
\end{tabular}
\vspace{0.5cm}

Menyatakan bahwa:
\begin{enumerate}
    \item Proyek pemrograman aplikasi (PROYEK 3) ini adalah karya asli yang belum pernah diajukan untuk memenuhi kelulusan pada program studi DIV Teknik Informatika di Universitas Logistik \& Bisnis Internasional maupun di perguruan tinggi lainnya.
    \item Proyek ini merupakan hasil pemikiran, rumusan, dan penelitian saya sendiri, tanpa adanya bantuan dari pihak lain, kecuali arahan yang diberikan oleh pembimbing.
    \item Dalam proyek ini, tidak terdapat karya atau pendapat yang ditulis atau dipublikasikan oleh orang lain, kecuali jika telah dicantumkan secara tertulis sebagai acuan dalam naskah.
    \item Saya menyatakan bahwa pernyataan ini dibuat dengan sesungguhnya.
\end{enumerate}

\vspace{1cm}
\begin{flushright}
    Bandung, Januari 2026\\
    Yang membuat pernyataan,\\[2cm]
    \textbf{Muhamad Haekal Syukur}\\
    NPM : 714230027
\end{flushright}

\chapter*{SURAT PERNYATAAN BEBAS PLAGIARISME}
\begin{tabular}{ll}
Nama & : Muhammad Ferdy Leoza \\
NPM & : 714230060 \\
Program Studi & : DIV Teknik Informatika \\
Judul & : NYUMBANGIN: PLATFORM DONASI DIGITAL
\end{tabular}
\vspace{0.5cm}

Menyatakan bahwa:
\begin{enumerate}
    \item Proyek pemrograman aplikasi (PROYEK 3) ini adalah karya asli yang belum pernah diajukan untuk memenuhi kelulusan pada program studi DIV Teknik Informatika di Universitas Logistik \& Bisnis Internasional maupun di perguruan tinggi lainnya.
    \item Proyek ini merupakan hasil pemikiran, rumusan, dan penelitian saya sendiri, tanpa adanya bantuan dari pihak lain, kecuali arahan yang diberikan oleh pembimbing.
    \item Dalam proyek ini, tidak terdapat karya atau pendapat yang ditulis atau dipublikasikan oleh orang lain, kecuali jika telah dicantumkan secara tertulis sebagai acuan dalam naskah.
    \item Saya menyatakan bahwa pernyataan ini dibuat dengan sesungguhnya.
\end{enumerate}

\vspace{1cm}
\begin{flushright}
    Bandung, Januari 2026\\
    Yang membuat pernyataan,\\[2cm]
    \textbf{Muhammad Ferdy Leoza}\\
    NPM : 714230060
\end{flushright}

\chapter*{ABSTRAK}
Aplikasi donasi digital kini banyak digunakan oleh kreator konten untuk memudahkan dukungan dari para pendukung. Laporan ini membahas pengembangan platform donasi dengan fitur utama seperti pengiriman donasi, notifikasi real-time melalui overlay, leaderboard pendukung, serta mekanisme pencairan dana bagi kreator. Pengembangan dilakukan melalui analisis kebutuhan, perancangan sistem, dan implementasi fitur sesuai alur donasi hingga pencairan dana. Hasil pengujian menunjukkan bahwa aplikasi dapat memproses donasi dengan baik, menampilkan notifikasi secara langsung dan menyediakan proses pencairan dana yang terstruktur untuk kreator. Secara keseluruhan, aplikasi yang dibangun telah memenuhi tujuan utama, yaitu menyediakan sarana donasi yang fungsional dan mudah digunakan, meskipun masih terdapat ruang pengembangan lebih lanjut untuk meningkatkan stabilitas dan cakupan fitur.

\vspace{0.5cm}
\textbf{Kata Kunci:} donasi digital, aplikasi web, kreator, notifikasi overlay, payout.

\chapter*{ABSTRACT}
\textit{Digital donations applications are now widely used by content creators to make easier for supporters to contribute. This report covers the development of a donation platform with key features such as donation delivery, real-time notifications via overlays, supporter leaderboards, and fund disbursement for creators. The development process included needs analysis, system design, and feature implementation following the donations flow up to payout. Testing show the applications can process donations well, display notifications instantly, and provide a structured fund disbursement process for creators. Overall, the applications that was built has met its main objective, which is to provide a functional and easy-to-use donation tool, although there is still room for further development to improve stability and feature coverage.}

\vspace{0.5cm}
\textbf{Keywords:} \textit{digital donations, web applications, creator, overlay notifications, payouts.}

\tableofcontents
\listoffigures
\listoftables

% --- MAIN CONTENT ---
% Wrapper file so bukped.tex (which includes chapters/introduction) finds Bab 1
% This simply inputs the existing bab1.tex
\input{chapters/bab1}
\input{chapters/bab2}

\chapter{LANDASAN TEORI}

\section{PLATFORM DONASI DIGITAL}
Platform donasi digital merupakan pengembangan dari teknologi platform berbasis internet yang memfasilitasi interaksi antara berbagai pihak untuk tujuan sosial dan filantropi. Platform digital didefinisikan sebagai seperangkat komponen teknologi yang menyediakan fungsi inti bagi suatu sistem dan menjadi fondasi bagi pengembangan layanan pelengkap di atasnya (Shneor et al., n.d.; Zhou \& Inoue, 2025). Secara konseptual, platform ini beroperasi sebagai \textit{two-sided market} yang mempertemukan kelompok pengguna berbeda namun saling bergantung, seperti donatur dan penerima manfaat, di mana nilai platform tercipta dari interaksi antar pengguna tersebut (Jullien et al., 2021).

Dalam konteks filantropi di Indonesia, platform digital digunakan sebagai alternatif lembaga amil konvensional dengan menawarkan kemudahan akses, transparansi, dan kecepatan distribusi dana untuk zakat, infaq, sedekah, dan wakaf bagi para donator (Febriandika et al., 2024; Hidayatullah \& Purbasari, 2022). Perkembangan ini sejalan dengan meningkatnya kepercayaan masyarakat terhadap sistem donasi daring serta pergeseran perilaku filantropi ke media digital (Indriyani \& Ibrahim, 2024).

Dalam lingkup penggalangan dana massal, platform \textit{donation-based crowdfunding} didefinisikan secara spesifik sebagai perantara teknologi yang digunakan oleh penggalang dana (\textit{fundraisers}) untuk mencocokkan donasi dengan tujuan para donatur. Tujuan yang dipertemukan dalam platform ini tidak bersifat moneter, melainkan pemenuhan kebutuhan psikologis donatur (Fadzirul Kamarubahrin et al., n.d.). Dalam ekosistem ini, penyedia situs web (\textit{website providers}) berperan menyediakan layanan web dan membangun sistem yang mendukung pemilik proyek (\textit{founders}) untuk mempresentasikan kampanye mereka kepada calon pendukung (\textit{backers}) guna menggalang donasi (Fadzirul Kamarubahrin et al., n.d.; Sirisawat et al., 2022).

\section{ARSITEKTUR APLIKASI WEB MODERN}
Bagian ini membahas konsep dasar arsitektur aplikasi web modern yang menjadi landasan dalam memahami cara kerja sistem berbasis web (Fraihat et al., 2022). Pemahaman mengenai pola komunikasi serta pembagian lapisan dalam aplikasi diperlukan untuk menjelaskan bagaimana komponen penyusun sistem saling berinteraksi dan menjalankan fungsinya. Oleh karena itu, pembahasan berikut difokuskan pada model \textit{client-server} dan arsitektur \textit{three-tier/N-tier} sebagai struktur arsitektural yang umum digunakan dalam pengembangan aplikasi web masa kini.

\subsection{Arsitektur Three-Tier/N-Tier}
Arsitektur \textit{three-tier} (tiga lapisan) atau N-tier merupakan sebuah model arsitektur perangkat lunak yang membagi fungsionalitas aplikasi menjadi tiga lapisan logis dan fisik yang berbeda untuk meningkatkan skalabilitas dan keandalan sistem (Maruf \& Ugli, n.d.; Prabu \& De Paul, 2025). Tiga lapisan utama tersebut terdiri dari: lapisan presentasi (\textit{presentation tier}), yang berinteraksi langsung dengan pengguna; lapisan aplikasi (\textit{application tier} atau \textit{business logic tier}), yang menangani pemrosesan data dan logika bisnis inti; dan lapisan data (\textit{data tier}), yang bertanggung jawab atas penyimpanan dan manajemen basis data (Maruf \& Ugli, n.d.). Pemisahan fungsionalitas ini memungkinkan setiap lapisan dikelola dan dikembangkan secara independen, menjadi arsitektur ini pilihan yang efektif untuk sistem yang memerlukan ketersediaan tinggi (\textit{high availability}), seperti pada kasus penerapan LMS Moodle (Ismail, 2023).

\subsection{Konsep Client-Server}
\textit{Client-Server} adalah sebuah model perangkat lunak yang memungkinkan sumber daya dan permintaan layanan dipenuhi melalui jaringan, dimana komputer yang disebut klien (\textit{client}) akan meminta sumber daya atau layanan, dan server akan menerima permintaan tersebut, memprosesnya, dan memberikan respons yang sesuai (Assistant professor, n.d.; Geofrey et al., n.d.). Model ini memungkinkan banyak pengguna untuk secara simultan mengakses dan menggunakan sumber daya yang disimpan secara terpusat di server, yang biasanya memiliki basis data dan menjalankan program untuk memproses permintaan (Geofrey et al., n.d.). Komunikasi antara klien dan server difasilitasi melalui protokol standar seperti HTTP, FTP, dan SMTP, dan model ini memberikan inter-proses komunikasi yang memungkinkan pertukaran data, menjadikannya fondasi bagi banyak aplikasi termasuk email, sistem basis data, dan internet (Nyabuto, 2023; Assistant professor, n.d.).

\section{REST API DAN PROTOKOL HTTP}
REST API dipahami sebagai pendekatan arsitektur \textit{web service} yang memanfaatkan prinsip \textit{Representational State Transfer} (REST) (Ehsan et al., 2022; Roziqin et al., 2023). REST menekankan penggunaan URI standar untuk mengidentifikasi \textit{resource}, memanfaatkan protokol dan prinsip yang sudah ada di web, serta menerapkan batasan seperti \textit{addressability}, \textit{statelessness}, \textit{uniform interface}, dan \textit{representations} untuk memastikan interaksi yang sederhana namun kuat dalam sistem terdistribusi (Ehsan et al., 2022). Dalam konteks ini, HTTP berperan sebagai protokol utama yang digunakan baik sebagai standar komunikasi maupun sebagai media transportasi data (Roziqin et al., 2023). HTTP memungkinkan klien dan server bertukar informasi melalui metode seperti GET, POST, PUT, dan DELETE, dan menjadi fondasi bagi layanan REST karena sifatnya yang terbuka, sederhana, serta telah lama menjadi protokol inti web modern (Apriyani \& Hamdana, n.d.; Roziqin et al., 2023).

\section{AUTENTIKASI DAN OTORISASI}
Bagian ini membahas konsep dasar autentikasi dan otorisasi yang menjadi fondasi penting dalam pengamanan aplikasi berbasis web. Mekanisme pengenalan identitas pengguna dan pemberian hak akses harus dirancang secara tepat agar interaksi antar sistem tetap aman, terukur, dan sesuai dengan tingkat kewenangan yang dibutuhkan. Oleh karena itu, pembahasan berikut mencakup OAuth 2.0 sebagai protokol delegasi akses, OpenID Connect sebagai lapisan identitas, JSON Web Token (JWT) sebagai format token yang umum digunakan, skema Bearer Token yang banyak diadopsi dalam komunikasi API, serta prinsip-prinsip keamanan API yang memastikan perlindungan terhadap ancaman dan penyalahgunaan akses.

\subsection{OAuth 2.0}
OAuth 2.0 didefinisikan sebagai \textit{framework} otorisasi yang populer yang memungkinkan suatu aplikasi memperoleh akses terbatas ke \textit{resource} yang dilindungi tanpa harus mengetahui atau menyimpan kredensial pengguna secara langsung (Lodderstedt et al., 2025). OAuth 2.0 menyediakan seperangkat \textit{authorization server} (Philippaerts et al., 2022). \textit{Framework} ini dirancang untuk mendukung berbagai konteks---mulai dari aplikasi web, \textit{single-page apps}, hingga aplikasi mobile---dengan cara memberikan fleksibilitas pada mekanisme autentikasi dan otorisasi yang aman di antara berbagai jenis klien (Singh \& Chaudhary, 2023).

\subsection{OpenID Connect}
OpenID Connect (OIDC) merupakan sebuah protokol yang mapan yang digunakan secara luas dalam manajemen identitas terfederasi (\textit{federated identity management}). Protokol ini berfungsi sebagai dasar bagi otentikasi dan sistem Masuk Tunggal (\textit{Single Sign-On} atau SSO), yang memungkinkan klien untuk memverifikasi identitas pengguna akhir berdasarkan otentikasi yang dilakukan oleh Penyedia Identitas (\textit{Identity Provider}) (Hammann et al., 2020; Yasuda et al., 2022). Dibangun di atas kerangka kerja otorisasi OAuth 2.0, kegunaan OIDC meluas hingga ke infrastruktur kompleks, seperti memfasilitasi akses Secure Shell (SSH) pada pengaturan terfederasi dengan menggunakan token akses OIDC untuk otentikasi pengguna pada server jarak jauh (Gudu et al., 2025).

\subsection{JWT}
JSON Web Token (JWT) merupakan sebuah standar terbuka yang didasarkan pada RFC 7519, yang digunakan secara luas sebagai mekanisme standar untuk otentikasi dan otorisasi pengguna pada layanan web. Standar ini tidak hanya populer untuk mengamankan transmisi data dan otentikasi pada RESTful API, tetapi juga dapat diperluas untuk meningkatkan keamanan dengan menyimpan informasi historis perilaku pengguna, seperti konsistensi alamat IP dan jenis \textit{user agent} (Bucko et al., 2023; Rahman et al., n.d.). Sementara itu, JWT secara fundamental adalah format token yang memfasilitasi transmisi data yang ringkas dan aman antara pihak-pihak yang berkepentingan sebagai objek JSON, yang menjadikannya mekanisme otentikasi yang penting dalam implementasi berbagai aplikasi modern (Nashikhuddin et al., 2023).

\subsection{Skema Bearer Token}
Skema Bearer Token merupakan mekanisme autentikasi pada OAuth 2.0 di mana klien cukup menyertakan token pada header (\texttt{Authorization: Bearer <token>}) untuk memperoleh akses ke \textit{resource} yang dilindungi (Lodderstedt et al., 2025). Karena token ini bersifat \textit{bearer}, siapa pun yang memilikinya dapat menggunakannya tanpa verifikasi tambahan, sehingga membuat keamanan transport menjadi aspek kritis. Penelitian terbaru menyoroti bahwa risiko pencurian token dapat diminimalkan melalui penggunaan kanal terenkripsi, pembatasan masa hidup token, serta validasi ketat pada sisi server (Ball et al., n.d.). Selain itu, praktik modern juga menekankan pentingnya menghindari pengiriman token melalui URL dan memastikan proses otorisasi mengikuti pedoman keamanan OAuth 2.0 (Neelan, 2022).

\subsection{Keamanan API}
Keamanan API merupakan aspek kritis dalam pengembangan aplikasi modern karena API menjadi jalur utama pertukaran data dan sering menjadi target serangan (Chandramouli \& Butcher, 2020). Banyak celah keamanan muncul akibat pengelolaan aset API yang lemah, API lama yang tidak terinventarisasi, serta kerentanan pada alur data dan logika bisnis (Sun et al., 2022). Selain itu, meningkatnya kompleksitas arsitektur RESTful dan GraphQL memperluas permukaan serangan, termasuk risiko seperti \textit{information leakage}, \textit{unauthorized access}, dan eksploitasi validasi input yang tidak memadai (Zhao, n.d.).

Untuk mengatasi ancaman tersebut, mekanisme keamanan API membutuhkan pendekatan berlapis yang mencakup autentikasi kuat berbasis OAuth/JWT, penggunaan HTTPS/TLS untuk mengamankan transmisi data, serta manajemen hak akses yang detail guna mencegah penyalahgunaan kredensial (Zhao, n.d.). Pentingnya teknik seperti \textit{asset discovery}, \textit{traffic auditing}, dan analisis alur data untuk mengidentifikasi API tersembunyi dan aktivitas mencurigakan (Sun et al., 2022). Di samping itu, penggunaan API Gateway dapat membantu menerapkan pembatasan trafik, filtrasi permintaan, dan perlindungan terhadap serangan seperti DDoS, sehingga API tetap terawasi dan terlindungi secara konsisten.

\section{DATABASE NoSQL (MongoDB)}
Basis data dokumen NoSQL (\textit{Not Only SQL}) muncul sebagai alternatif yang signifikan terhadap basis data relasional tradisional yang sering memiliki batasan ketat pada struktur data dan relasi, sehingga kurang efisien untuk menangani volume data yang sangat besar (\textit{huge database}) (Byali et al., 2022). NoSQL document database mengatasi masalah ini dengan menyediakan kemampuan untuk menyimpan dan mengelola data dalam format dokumen, sehingga dapat menampung data yang tidak terstruktur, semi-struktur, maupun terstruktur (Carvalho et al., 2023). Keunggulan utama NoSQL, khususnya jenis berorientasi dokumen seperti MongoDB, terletak pada fleksibilitas dan skalabilitas horizontal yang tinggi, menjadikannya pilihan esensial ketika skema data yang dinamis tidak sesuai dengan kebutuhan basis data relasional (Byali et al., 2022).

\section{FLOWCHART SISTEM}
Flowchart merupakan salah satu model yang paling mendasar dan penting dalam perancangan sistem informasi, di mana ia berfungsi untuk mendesain dan menyusun alur dokumen serta memvisualisasikan prosedur atau tahapan proses secara sistematis. Secara umum, flowchart memiliki aplikasi yang luas di berbagai bidang seperti pengembangan perangkat lunak, desain teknik, dan eksperimen ilmiah (Zhang et al., 2023). Struktur data flowchart tradisional sering kali didasarkan pada \textit{adjacency list}, \textit{cross-linked list}, atau \textit{adjacency matrix} dari struktur graf, yang didasari fakta bahwa setiap dua node dapat memiliki hubungan koneksi (Zhang et al., 2023). Namun, terlepas dari kompleksitas penyimpanan datanya, flowchart tetap menjadi alat fundamental yang menyediakan representasi visual dari urutan dan hubungan logis dalam suatu sistem (Ratumurun et al., n.d.; Pan et al., 2024).

\section{UNIFIED MODELING LANGUAGE (UML)}
Unified Modeling Language (UML) didefinisikan sebagai sebuah bahasa pemodelan standar yang digunakan untuk merancang dan mendokumentasikan sistem berorientasi objek. Sebagai bahasa standar, UML menyediakan seperangkat notasi grafis yang komprehensif untuk memvisualisasikan, menspesifikasikan, membangun, dan mendokumentasikan artefak dalam sistem perangkat. Tujuan utama penggunaan UML adalah untuk memperjelas model yang tidak konsisten dan mengurangi ambiguitas selama proses pengembangan perangkat lunak (Amani Bestari et al., 2024). UML membantu memvisualkan, menspesifikasikan dan mendokumentasikan desain sistem secara grafis (Siska Narulita et al., 2024).

Dengan menggunakan diagram-diagram yang berbeda, seperti Use Case Diagram dan Activity Diagram, UML membantu pengembang dalam memodelkan interaksi, struktur, dan perilaku sistem (Marwah M. A. Dabdawb, 2024). Penerapan UML sangat krusial dalam siklus hidup pengembangan sistem (\textit{System Development Life Cycle} atau SDLC) karena membantu memastikan konsistensi model dan mempermudah komunikasi antara pihak-pihak yang terlibat dalam proyek (Marchezan et al., 2023).

\subsection{Use Case Diagram}
Use Case adalah suatu diagram fundamental yang umum diajarkan dalam ilmu komputer dan rekayasa perangkat lunak. Diagram ini berfungsi sebagai representasi visual dari fungsionalitas sistem dari sudut pandang pengguna. Meskipun definisinya tampak sederhana, penilaian terhadap diagram use case sering kali menjadi hambatan dalam proses pembelajaran, terutama karena dua masalah utama: masalah interpersonal (tidak adanya standar penilaian di antara para pengajar) (Jebli et al., 2024) dan masalah intrapersonal (inkonsistensi seorang pengajar saat menilai banyak diagram) (Fauzan et al., 2021; Abbott et al., 2025; Wang et al., 2025).

\subsection{Activity Diagram}
Activity Diagram adalah salah satu diagram perilaku yang tersedia dalam Unified Model Language (UML) yang digunakan untuk memodelkan alur kontrol dan alur data dalam suatu sistem (Sandfreni et al., 2021). Diagram ini secara visual merepresentasikan langkah-langkah, keputusan, dan urutan tindakan yang diperlukan untuk menyelesaikan suatu proses atau kegiatan bisnis tertentu (Siska Narulita et al., 2024). Dalam konteks pemodelan sistem, Activity Diagram sangat berguna untuk memvisualisasikan bagaimana berbagai kegiatan saling terkait dan bergantung satu sama lain (Jha et al., 2023; Ramdany et al., n.d.).

\subsection{Sequence Diagram}
Sequence Diagram adalah diagram UML yang paling umum kedua, digunakan untuk merepresentasikan interaksi objek dan pertukaran pesan antar objek tersebut seiring berjalannya waktu (Siska Narulita et al., 2024). Diagram ini secara visual menunjukkan bagaimana peristiwa atau aktivitas yang ada dalam sebuah use case dipetakan menjadi operasi-operasi dari kelas objek yang ada pada Class Diagram (Al-Fedaghi, n.d.).

\subsection{Class Diagram}
Class Diagram merupakan salah satu diagram Unified Modeling Language (UML) yang paling umum digunakan dalam pendidikan dan pengembangan perangkat lunak berorientasi objek (Siska Narulita et al., 2024b). Fungsi utama dari Class Diagram adalah untuk merepresentasikan kelas-kelas dalam sistem perangkat lunak dan hubungan yang terjalin antar kelas-kelas tersebut (Fauzan et al., 2021b).

\section{ENTITY-RELATIONSHIP DIAGRAM (ERD)}
Entity-Relationship Diagram (ERD) adalah salah satu teknik utama yang digunakan dalam perancangan basis data dan merupakan representasikan diagramatik utama dari model data konseptual (Pulungan et al., n.d.). Fungsi utamanya adalah untuk merefleksikan persyaratan data pengguna dalam suatu sistem basis data (PENGANTAR BASIS DATA, n.d.). ERD adalah tahap pertama dalam proses desain basis data dan memvisualisasikan bagaimana berbagai komponen data diatur dan berinteraksi (Jaimez-González \& Martínez-Samora, 2020). Dalam membuat ERD, beberapa hal penting harus dipertimbangkan, antara lain setiap basis data harus memiliki entitas (\textit{entities}) yang saling terhubung melalui hubungan (\textit{relationship}), dan setiap entitas harus memiliki atribut (\textit{attributes}) yang terdiri dari kunci utama (\textit{primary key}) dan kunci asing (\textit{foreign key}) (Afiifah et al., 2022).

\section{PERAN FITUR SOCIAL SHARING DALAM DONASI DIGITAL}
Kepercayaan merupakan faktor kunci dalam keberhasilan platform donasi digital, mengingat interaksi antara donatur dan penerima dilakukan sepenuhnya secara daring tanpa kontak langsung (Tarigan, 2023). Tingkat kepercayaan ini sangat dipengaruhi oleh persepsi transparansi dan kredibilitas sistem yang disediakan oleh platform donasi digital (Greselda Gosal et al., n.d.). Oleh karena itu, platform donasi perlu menyediakan mekanisme yang mampu memperkuat transparansi tersebut.

Salah satu mekanisme yang dapat diterapkan adalah fitur \textit{social sharing}, yang memungkinkan aktivitas donasi dibagikan ke media sosial. Penerapan fitur \textit{social sharing} berperan sebagai \textit{social proof}, di mana visibilitas partisipasi pengguna lain dapat meningkatkan persepsi keandalan dan legitimasi platform (Rahmayanti et al., 2024). Peningkatan kepercayaan dan persepsi positif tersebut pada akhirnya berdampak pada meningkatnya niat dan partisipasi pengguna dalam kegiatan donasi digital (Kamarudin et al., 2023).

Berdasarkan temuan tersebut, fitur \textit{social sharing} pada aplikasi Nyumbangin dirancang sebagai fitur pendukung yang memungkinkan pengguna membagikan aktivitas donasi ke media sosial. Fitur ini diharapkan dapat memperkuat kepercayaan pengguna serta mendorong partisipasi donasi secara lebih luas.

\chapter{METODE PENELITIAN}

\section{METODE PENGEMBANGAN SISTEM}
Pengembangan platform Nyumbangin menggunakan pendekatan \textit{Agile}, yang menekankan proses pengembangan sistem secara bertahap, adaptif, dan berulang. Pendekatan \textit{Agile} dipilih karena sesuai dengan karakteristik proyek berskala kecil hingga menengah, serta memungkinkan penyesuaian fitur berdasarkan hasil evaluasi pada setiap tahap pengembangan.

Pendekatan ini memungkinkan sistem dikembangkan secara inkremental, di mana setiap iterasi menghasilkan fungsionalitas yang dapat diuji dan dievaluasi sebelum melanjutkan ke iterasi berikutnya. Dengan demikian, risiko kesalahan desain dan implementasi dapat diminimalkan sejak tahap awal.

\subsection{Konsep Agile Development}
\textit{Agile Development} merupakan pendekatan pengembangan perangkat lunak yang berfokus pada fleksibilitas, kolaborasi, dan kemampuan beradaptasi terhadap perubahan kebutuhan (Ariesta et al., n.d.). Berbeda dengan metode linear seperti \textit{waterfall}, \textit{Agile} memungkinkan perubahan kebutuhan terjadi selama proses pengembangan tanpa harus mengulang seluruh tahapan dari awal.

Dalam konteks pengembangan platform Nyumbangin, \textit{Agile} digunakan sebagai kerangka kerja konseptual untuk mengelola proses pengerjaan fitur secara bertahap, mulai dari analisis kebutuhan dasar, implementasi modul inti, hingga pengujian dan evaluasi sistem. Pendekatan ini mendukung pengembangan sistem yang responsif terhadap kebutuhan fungsional dan teknis yang berkembang selama proyek berlangsung.

\subsection{Alur Iterasi Pengembangan}
Proses pengembangan sistem dilakukan melalui beberapa siklus iterasi yang masing-masing mencakup tahapan:
\begin{enumerate}
    \item \textbf{Perencanaan Iterasi}: Penentuan fitur yang akan dikembangkan berdasarkan prioritas kebutuhan sistem.
    \item \textbf{Implementasi Fitur}: Pengembangan modul atau fungsi tertentu sesuai dengan hasil perencanaan iterasi.
    \item \textbf{Pengujian Fungsional}: Pengujian terhadap fitur yang telah dikembangkan untuk memastikan kesesuaian dengan kebutuhan.
    \item \textbf{Evaluasi dan Penyempurnaan}: Evaluasi hasil iterasi dan perbaikan terhadap kekurangan sebelum melanjutkan ke iterasi berikutnya.
\end{enumerate}
Setiap iterasi menghasilkan peningkatan fungsional sistem yang dapat langsung diuji, sehingga kemajuan proyek dapat dipantau secara berkelanjutan.

\subsection{Penerapan Agile pada Proyek Nyumbangin}
Penerapan pendekatan \textit{Agile} pada proyek Nyumbangin dilakukan dengan membagi pengembangan sistem ke dalam beberapa iterasi utama. Iterasi awal difokuskan pada pembangunan fitur inti, seperti autentikasi kreator, pencatatan donasi, dan integrasi sistem pembayaran. Iterasi berikutnya mencakup pengembangan fitur pendukung, seperti notifikasi \textit{real-time} melalui overlay, leaderboard donatur, serta mekanisme \textit{payout} bagi kreator.

Pada setiap iterasi, fitur yang telah diimplementasikan langsung diuji menggunakan skenario pengujian fungsional untuk memastikan alur donasi, pembayaran, dan pencairan dana berjalan dengan benar. Hasil pengujian digunakan sebagai dasar evaluasi untuk menentukan perbaikan atau pengembangan fitur pada iterasi selanjutnya. Pendekatan ini memungkinkan pengembangan aplikasi dilakukan secara terstruktur namun tetap fleksibel, sehingga sistem dapat berkembang secara bertahap hingga memenuhi kebutuhan fungsional yang telah ditetapkan.

\section{ANALISIS KEBUTUHAN}

\subsection{Sumber Kebutuhan}
Analisis kebutuhan sistem dilakukan melalui tiga pendekatan utama. Pertama, observasi terhadap platform donasi digital untuk mengidentifikasi pola umum, seperti kebutuhan transparansi transaksi, tampilan notifikasi \textit{real-time}, dan mekanisme \textit{payout} yang akuntabel. Kedua, studi pustaka terkait autentikasi modern (OAuth 2.0, JWT), keamanan API, serta pola desain sistem web yang relevan dengan karakteristik aplikasi donasi. Ketiga, analisis artefak kode dan struktur API yang telah dikembangkan, termasuk model basis data, alur donasi, integrasi Midtrans, serta skrip pemeliharaan yang digunakan untuk verifikasi dan pengarsipan data. Pendekatan ini memastikan kebutuhan sistem dirumuskan berdasarkan konteks teknis dan operasional yang aktual.

\subsection{Kebutuhan Fungsional}
Kebutuhan fungsional mendeskripsikan fitur yang wajib disediakan agar aplikasi donasi dapat berfungsi secara utuh. Fitur tersebut meliputi:
\begin{enumerate}
    \item \textbf{Autentikasi Pengguna} menggunakan Google OAuth melalui NextAuth, serta validasi akses melalui JWT untuk endpoint sensitif.
    \item \textbf{Validasi Username Kreator}, memastikan username unik dan dapat diverifikasi sebelum transaksi dilakukan.
    \item \textbf{Pengelolaan Donasi}, mencakup pembuatan transaksi, integrasi Midtrans, pembaruan status melalui webhook, serta penyimpanan konten media share.
    \item \textbf{Notifikasi Real-Time}, yang menampilkan donasi terbaru pada overlay kreator untuk keperluan siaran langsung.
    \item \textbf{Leaderboard}, baik secara global maupun berdasarkan kreator, untuk menampilkan agregasi donasi.
    \item \textbf{Statistik Kreator}, berupa ringkasan donasi berdasarkan periode.
    \item \textbf{Payout}, mencakup permintaan penarikan dana, perhitungan biaya platform, dan persetujuan admin.
    \item \textbf{Pengelolaan Data Media Share dan Notifikasi}, termasuk pengaturan masa berlaku (TTL) dan keterkaitan dengan transaksi donasi.
\end{enumerate}

\subsection{Kebutuhan Non-Fungsional}
Kebutuhan non-fungsional mencakup karakteristik kualitas sistem, yaitu:
\begin{itemize}
    \item \textbf{Keamanan}: verifikasi token JWT, sanitasi input, pembatasan metode HTTP, dan pemisahan akses berdasarkan peran pengguna.
    \item \textbf{Performa}: optimasi query leaderboard melalui limit dan sorting.
    \item \textbf{Skalabilitas}: rencana pagination serta caching pada proses agregasi data.
    \item \textbf{Reliabilitas}: konsistensi penanganan kesalahan dengan kode status standar (401, 404, 500).
    \item \textbf{Integritas Data}: akurasi perhitungan saldo payout dan pemrosesan donasi berdasarkan status valid (PAID).
\end{itemize}

\section{PERANCANGAN SISTEM}

\subsection{Arsitektur Logis}
Arsitektur logis sistem terdiri atas empat lapisan utama:
\begin{enumerate}
    \item \textbf{Lapisan Antarmuka Pengguna (UI Layer)}: Berisi halaman donasi, dashboard kreator, halaman overlay notifikasi, serta antarmuka admin.
    \item \textbf{Lapisan API (Application Layer)}: Mengelola endpoint untuk donasi, leaderboard, overlay, autentikasi, payout, dan operasi admin melalui mekanisme API Routes di Next.js.
    \item \textbf{Lapisan Layanan Utilitas (Service Layer)}: Meliputi modul koneksi database, pengelolaan token JWT, serta utilitas untuk validasi dan perhitungan internal.
    \item \textbf{Lapisan Data (Data Layer)}: Terdiri atas model MongoDB seperti Donation, Creator, MediaShare, Payout, dan Notification.
\end{enumerate}

\subsection{Arsitektur Fisik}
Arsitektur fisik sistem mengikuti pola aplikasi web modern:
Browser / OBS Overlay $\rightarrow$ Next.js Runtime (Node.js) $\rightarrow$ MongoDB $\rightarrow$ Layanan Eksternal (Google OAuth, Midtrans).
\begin{itemize}
    \item Next.js menangani logika UI dan API dalam satu platform.
    \item MongoDB digunakan sebagai basis data dokumen.
    \item Midtrans mengelola proses pembayaran melalui Snap Token dan webhook.
    \item Overlay digunakan secara mandiri melalui OBS atau iframe untuk menampilkan notifikasi donasi.
\end{itemize}

\subsection{Arsitektur Teknologi}
Arsitektur teknologi sistem mencakup penggunaan Next.js sebagai framework utama yang menjalankan frontend and backend melalui API Routes, Node.js sebagai runtime \textit{server-side}, serta MongoDB sebagai data dokumen. Sistem autentikasi menggunakan Google OAuth 2.0 melalui NextAuth dan JWT untuk otorisasi endpoint privat. Mekanisme pembayaran dilakukan dengan Midtrans melalui Snap Token dan Webhook. Selain itu, aplikasi menyediakan overlay web \textit{real-time} untuk integrasi dengan OBS sebagai tampilan notifikasi donasi. Kombinasi teknologi ini menghasilkan sistem yang modern, responsif, serta siap diintegrasikan dengan berbagai layanan eksternal.

\begin{figure}[htbp]
    \centering
    \includegraphics[width=0.8\textwidth]{images/arsitektur_teknologi.png}
    \caption{Arsitektur Teknologi}
    \label{fig:arsitektur_teknologi}
\end{figure}

\subsection{Modul Utama}
Sistem dibagi ke dalam beberapa modul utama:
\begin{enumerate}
    \item Authentication Module (OAuth + JWT)
    \item Donation Module (pembuatan transaksi, webhook, media share)
    \item Leaderboard Module (global dan per kreator)
    \item Payout Module (request, approval, perhitungan fee)
    \item Notification Module (TTL, event donasi)
    \item Maintenance Module (arsip donasi, integritas data)
\end{enumerate}

\subsection{Strategi Desain}
Desain strategi menerapkan pola penanganan API yang konsisten meliputi validasi metode HTTP, autentikasi, validasi input, eksekusi query database, dan pengembalian response JSON. Selain itu, prinsip pemisahan tanggung jawab diterapkan melalui pembagian endpoint berdasarkan role dan fungsi. Pembatasan data seperti limit dan sorting digunakan untuk menghindari \textit{over-fetching}, terutama pada leaderboard dan statistik.

\section{PEMODELAN SISTEM}
Pemodelan sistem dilakukan untuk memberikan representasi visual dan konseptual dari fungsi, alur kerja, serta struktur data yang digunakan dalam aplikasi Nyumbangin ini. Pemodelan ini bertujuan memastikan bahwa kebutuhan fungsional dan non-fungsional yang telah diidentifikasi dapat diterjemahkan ke dalam desain sistem yang jelas, terstruktur, dan mudah diimplementasikan. Diagram-diagram pada bagian ini mencakup model proses, interaksi, dan entitas yang saling berhubungan, sehingga dapat memberikan gambaran menyeluruh mengenai cara sistem beroperasi secara \textit{end-to-end}.

\subsection{Use Case}
Use case menggambarkan interaksi antara Donatur, Kreator, Admin, Midtrans, dan Google OAuth, yang meliputi proses donasi, verifikasi status pembayaran, pengelolaan leaderboard, permintaan payout, pengelolaan payout admin, dan login menggunakan Google OAuth.

\begin{figure}[htbp]
    \centering
    \includegraphics[width=0.8\textwidth]{images/usecase_diagram.png}
    \caption{Use Case Diagram}
    \label{fig:usecase_diagram}
\end{figure}

\subsection{Activity Diagram}
\textbf{Activity Diagram proses Donasi}:
Activity diagram proses Donasi menggambarkan di mana Donor mengisi formulir (dengan atau tanpa mediashare), kemudian Sistem membuat record PENDING, menghasilkan snap token, dan mengarahkan Donor ke Midtrans untuk pembayaran; setelah Midtrans mengirim webhook, Sistem memperbarui status donasi menjadi PAID dan membuat notifikasi.

\begin{figure}[htbp]
    \centering
    \includegraphics[width=0.8\textwidth]{images/activity_donasi.png}
    \caption{Activity Diagram Donasi}
    \label{fig:activity_donasi}
\end{figure}

\textbf{Activity Diagram proses Payout}:
Activity diagram proses Payout dimulai ketika Kreator meminta payout, di mana Sistem memeriksa saldo minimal; jika memenuhi syarat, sistem menghitung biaya layanan (5\%) dan status menjadi PENDING, lalu Admin meninjau dan melakukan transfer manual, yang kemudian diperbarui oleh Sistem menjadi PROCESSED.

\begin{figure}[htbp]
    \centering
    \includegraphics[width=0.8\textwidth]{images/activity_payout.png}
    \caption{Activity Diagram Payout}
    \label{fig:activity_payout}
\end{figure}

\subsection{Sequence Diagram}
\textbf{Sequence Diagram Donasi}:
Pada sequence diagram ini menggambarkan alur proses donasi, dimulai dari donor mengisi form di Donation Page yang kemudian divalidasi dan dikirim ke Donate API. Setelah validasi server-side, data donasi disimpan ke database, dan jika ada youtubeUrl, dibuat juga MediaShare. API lalu meminta Snap Token ke Midtrans, yang digunakan untuk membuka halaman pembayaran. Setelah donor membayar dan pembayaran sukses, Midtrans mengirim webhook ke server, sehingga status donasi diupdate menjadi PAID dan di akhir notifikasi overlay dibuat.

\begin{figure}[htbp]
    \centering
    \includegraphics[width=0.8\textwidth]{images/sequence_donasi.png}
    \caption{Sequence Diagram Donasi}
    \label{fig:sequence_donasi}
\end{figure}

\textbf{Sequence Diagram Leaderboard}:
Sequence diagram ini menggambarkan alur pengambilan data leaderboard, di mana client (dashboard kreator) mengirim request ke leaderboard API dengan token. API memeriksa method dan validasi token melalui JWT Service, lalu memastikan user adalah kreator dan datanya ada. Setelah itu, API mengambil data donasi terbaru dari database, memformat respons, dan mengirim hasilnya ke client. Jika terjadi error seperti token tidak valid, kreator tidak ditemukan, atau method salah, API akan mengembalikan kode error yang sesuai.

\begin{figure}[htbp]
    \centering
    \includegraphics[width=0.8\textwidth]{images/sequence_leaderboard.png}
    \caption{Sequence Diagram Leaderboard}
    \label{fig:sequence_leaderboard}
\end{figure}

\subsection{Model Entitas}
Model entitas digunakan untuk merepresentasikan struktur data utama yang bekerja di dalam platform donasi. Setiap entitas dirancang untuk mendukung proses transaksi, pengelolaan kreator, penayangan media share di overlay, hingga alur pencairan dana. Secara umum, entitas yang digunakan dapat dikelompokkan menjadi entitas utama dan entitas pendukung.

Entitas utama platform meliputi:
\begin{itemize}
    \item \textbf{Creator}: Menyimpan data kreator seperti nama, email, profil, serta informasi akun yang diperlukan untuk menerima donasi dan melakukan permintaan payout.
    \item \textbf{Donation}: Mencatat seluruh transaksi donasi, termasuk nominal, pesan, metode pembayaran, status (PENDING/PAID), serta relasi terhadap kreator yang menerima donasi.
    \item \textbf{MediaShare}: Entitas untuk menangani request media share (youtube video) yang dikaitkan dengan donasi tertentu, termasuk durasi dan validasi media.
    \item \textbf{Payout}: Menyimpan informasi permintaan pencairan dana kreator, mencakup jumlah pencairan, fee platform, status (PENDING/APPROVED/PROCESSED), serta log aktivitas admin.
    \item \textbf{Notification}: Berfungsi untuk menampilkan data overlay secara real-time, seperti donasi terbaru atau media share yang harus ditayangkan oleh streamer/kreator.
\end{itemize}

Selain itu, terdapat entitas pendukung yang digunakan untuk historisasi dan agregasi data:
\begin{itemize}
    \item \textbf{DonationHistory}: mencatat perubahan status donasi.
    \item \textbf{MonthlyLeaderboard}: digunakan untuk menyimpan data peringkat donatur setiap bulan.
    \item \textbf{Contact}: mencatat umpan balik dari pengguna.
    \item \textbf{ProfileImage}: menyimpan data gambar untuk kebutuhan profil kreator.
    \item \textbf{Admin}: menyimpan kredensial admin yang bertugas memverifikasi payout dan melakukan manajemen sistem.
\end{itemize}
Seluruh entitas tersebut berperan dalam memastikan integritas data serta menghubungkan seluruh proses inti mulai dari transaksi donasi, pengelolaan kreator, hingga operasional sistem admin.

\subsection{Relasi Entitas}
Pada poin ini menjelaskan hubungan antar entitas utama yang digunakan dalam sistem. Relasi ini dibangun berdasarkan alur operasional aplikasi, seperti proses donasi, pemutaran media share, pengajuan payout, serta notifikasi kepada kreator. Hubungan antar entitas divisualisasikan dalam bentuk Entity Relationship Diagram (ERD) agar struktur data menjadi lebih jelas, baik dari sisi keterkaitan maupun dependensi antar tabel/model. Diagram ini menjadi dasar dalam perancangan database dan memastikan bahwa setiap proses bisnis memiliki representasi data yang konsisten dan saling terhubung.

\begin{figure}[htbp]
    \centering
    \includegraphics[width=0.8\textwidth]{images/erd.png}
    \caption{ERD}
    \label{fig:erd}
\end{figure}

\section{METODE PERANCANGAN TEKNIS}
Perancangan teknis pada platform donasi ini difokuskan pada penyusunan arsitektur layanan yang aman, efisien, dan mudah dipelihara. Pendekatan utama yang digunakan adalah pemisahan tanggung jawab antar modul serta penerapan pola penanganan API yang konsisten. Setiap endpoint dirancang mengikuti alur standar: validasi metode HTTP, autentikasi menggunakan JWT (untuk endpoint privat), validasi input, eksekusi operasi database, dan pengembalian respons JSON. Pola yang seragam ini memudahkan debugging serta menjaga konsistensi perilaku lintas layanan.

Dari sisi keamanan, validasi token JWT diterapkan untuk memastikan setiap permintaan memiliki otorisasi yang benar termasuk pengecekan masa berlaku token dan jenis pengguna (donor, kreator, atau admin). Seluruh input kritis seperti username, nominal donasi, dan URL media share – validasi untuk mencegah data tidak sah masuk ke sistem.

Integrasi pembayaran dirancang agar bergantung hanya pada webhook resmi Midtrans, sehingga status transaksi tidak bergantung pada aktivitas pengguna di sisi client. Optimasi basis data dilakukan melalui penempatan indeks pada atribut yang sering digunakan dalam kueri seperti (createdAt, creatorId, dan creatorUsername). Operasi agregasi seperti leaderboard dan statistik kreator menggunakan pipeline agregasi MongoDB untuk mengurangi beban pada server aplikasi. Pembatasan kueri (limit) diterapkan untuk mencegah pengambilan data berlebihan pada endpoint yang memiliki potensi pertumbuhan data tinggi.

Selain itu, proses donasi dan penyajian notifikasi dipisahkan dari alur pembayaran utama. Server hanya membuat record donasi dan Snap Token pada permintaan awal, sementara pembaruan status dan pemicu notifikasi dilakukan ketika webhook diterima. Pemisahan ini membuat sistem lebih stabil dan memastikan overlay selalu menampilkan data yang sudah tervalidasi.

\section{METODE PENGUJIAN}
Metode pengujian pada platform donasi ini dirancang untuk memastikan bahwa seluruh fungsi sistem berjalan sesuai kebutuhan, aman digunakan, dan menghasilkan data yang konsisten. Pendekatan pengujian dilakukan melalui kombinasi pengujian unit, pengujian integrasi, serta pengujian fungsional terhadap endpoint API dan alur bisnis utama. Fokus utama pengujian meliputi keakuratan proses donasi, keandalan mekanisme payout, integritas data pada leaderboard serta statistik kreator, dan validitas proses autentikasi berbasis OAuth dan JWT.

Pengujian dilakukan menggunakan data uji yang dikontrol, simulasi webhook pembayaran, serta verifikasi hasil langsung melalui database. Seluruh skenario kritis seperti validasi input, autentikasi dan otorisasi, serta penanganan kesalahan diuji untuk memastikan sistem tetap stabil dalam berbagai kondisi operasional.

\subsection{Jenis Pengujian}
Pengujian sistem mencakup beberapa jenis pengujian sebagai berikut:
\begin{enumerate}
    \item \textbf{Pengujian Unit (Unit Testing)}: Dilakukan pada fungsi atau modul kecil yang berdiri sendiri, seperti validasi token JWT, perhitungan \textit{platformFee}, dan validasi input donasi.
    \item \textbf{Pengujian Integrasi (Integration Testing)}: Berfokus pada alur yang melibatkan beberapa komponen, seperti proses donasi lengkap, permintaan payout, dan penampilan notifikasi overlay.
    \item \textbf{Pengujian Fungsional (Functional Testing)}: Dilakukan untuk mengevaluasi apakah setiap endpoint memenuhi kebutuhan fungsional yang telah ditetapkan.
    \item \textbf{Pengujian Keamanan (Security Testing)}: Meliputi akses API tanpa token, token expired, dan payload tidak valid.
    \item \textbf{Pengujian Kinerja (Performance Testing)}: Difokuskan pada kecepatan query, respons webhook, dan performa overlay.
\end{enumerate}

\subsection{Skenario Pengujian}
Beberapa skenario pengujian utama yang digunakan meliputi:
\begin{enumerate}
    \item \textbf{Skenario 1 – Donasi Berhasil}: Input valid $\rightarrow$ server membuat record PENDING $\rightarrow$ Snap Token sukses $\rightarrow$ pembayaran di Midtrans $\rightarrow$ webhook diterima $\rightarrow$ status menjadi PAID $\rightarrow$ overlay menampilkan donasi.
    \item \textbf{Skenario 2 – Donasi Gagal Validasi}: Nominal $<$ minimum atau format URL salah $\rightarrow$ server mengembalikan status 400.
    \item \textbf{Skenario 3 – Akses Endpoint Tanpa Token}: Mengakses leaderboard atau payout tanpa Authorization $\rightarrow$ harus menghasilkan 401 Unauthorized.
    \item \textbf{Skenario 4 – Payout Request oleh Kreator}: Saldo cukup $\rightarrow$ request dicatat $\rightarrow$ status PENDING $\rightarrow$ admin review $\rightarrow$ APPROVED $\rightarrow$ status PROCESSED $\rightarrow$ saldo kreator berkurang.
    \item \textbf{Skenario 5 – Token Salah Atau Expired}: Token invalid/expired $\rightarrow$ akses ditolak dengan pesan error yang konsisten.
    \item \textbf{Skenario 6 – Webhook Simulasi Midtrans}: Webhook dikirim manual dari Postman $\rightarrow$ status donasi harus berubah menjadi PAID $\rightarrow$ overlay menampilkan notifikasi.
\end{enumerate}

\subsection{Alat Pengujian}
Alat yang digunakan dalam proses pengujian meliputi:
\begin{itemize}
    \item \textbf{Postman}: untuk pengujian API (header, body, autentikasi).
    \item \textbf{MongoDB Compass}: untuk memverifikasi perubahan data secara langsung.
    \item \textbf{Logging Next.js (console)}: untuk memantau webhook, error, dan alur proses.
\end{itemize}

\section{EVALUASI KEBERHASILAN}
Evaluasi keberhasilan dilakukan untuk menilai sejauh mana implementasi sistem memenuhi kebutuhan fungsional, stabilitas operasional, serta ketepatan mekanisme kritis seperti autentikasi, pengelolaan sesi, dan pemrosesan pembayaran. Penilaian dilakukan melalui pengujian terstruktur dan analisis hasil \textit{code coverage} yang dihasilkan dari proses \textit{unit testing} pada modul-modul inti.

\subsection{Interpretasi dan Evaluasi}
Berdasarkan hasil pengujian dan \textit{code coverage}, dapat disimpulkan bahwa:
\begin{enumerate}
    \item Stabilitas alur bisnis utama sudah terverifikasi, terutama mekanisme donasi dan payout yang melibatkan transaksi dan webhook.
    \item Keamanan dasar terkait autentikasi, JWT, dan sesi sudah diuji dan berfungsi sesuai kebutuhan.
    \item Konsistensi data pada proses pembayaran serta pencatatan notifikasi berhasil diuji melalui simulasi webhook.
    \item Coverage yang rendah lebih disebabkan oleh lingkup pengujian yang difokuskan, bukan karena ketidakterujian seluruh sistem.
    \item Sistem dinilai layak digunakan, namun peningkatan cakupan pengujian tetap direkomendasikan untuk modul non-kritis seperti UI dan utilitas.
\end{enumerate}

\subsection{Kesimpulan Evaluasi}
Secara keseluruhan, sistem telah memenuhi fungsi utamanya – mulai dari pemrosesan donasi, pembayaran, hingga payout dan notifikasi. Hasil pengujian menunjukkan bahwa fitur inti berjalan stabil, meskipun pengujian yang lebih luas masih diperlukan pada pengembangan selanjutnya.

\chapter{HASIL DAN PEMBAHASAN}

\section{HALAMAN AUTENTIKASI}
Berisi halaman yang menampilkan dua opsi untuk login. Login manual maupun menggunakan Google OAuth untuk mempermudah proses autentikasi.

\begin{figure}[htbp]
    \centering
    \includegraphics[width=0.8\textwidth]{images/halaman_login.png}
    \caption{Halaman Login}
    \label{fig:halaman_login}
\end{figure}

\section{HALAMAN DONASI}
\subsection{Form Donasi}
Halaman donasi menyediakan input seperti nama donor, nominal donasi, pesan serta opsi media share. Sistem melakukan validasi dasar sebelum melanjutkan proses ke Midtrans.

\begin{figure}[htbp]
    \centering
    \includegraphics[width=0.8\textwidth]{images/form_donasi.png}
    \caption{Form Donasi}
    \label{fig:form_donasi}
\end{figure}

\subsection{Halaman Pembayaran Midtrans}
Setelah form donasi divalidasi, sistem mengirimkan request Snap Token ke Midtrans. Donor kemudian diarahkan ke halaman pembayaran yang disediakan Midtrans.

\begin{figure}[htbp]
    \centering
    \includegraphics[width=0.8\textwidth]{images/pembayaran_midtrans.png}
    \caption{Halaman Pembayaran Midtrans}
    \label{fig:pembayaran_midtrans}
\end{figure}

\section{DASHBOARD KREATOR}
Pada dashboard kreator menampilkan statistik utama seperti total donasi, total pendapatan, menu riwayat donasi, dan menu leaderboard donasi terbanyak perbulan.

\begin{figure}[htbp]
    \centering
    \includegraphics[width=0.8\textwidth]{images/dashboard_kreator.png}
    \caption{Dashboard Kreator}
    \label{fig:dashboard_kreator}
\end{figure}

\section{HALAMAN REQUEST PAYOUT}
Kreator dapat mengajukan pencairan dana berdasarkan saldo yang tersedia. Sistem menghitung saldo bersih (total donasi PAID yang belum pernah dipayout).

\begin{figure}[htbp]
    \centering
    \includegraphics[width=0.8\textwidth]{images/halaman_payout.png}
    \caption{Halaman Payout}
    \label{fig:halaman_payout}
\end{figure}

\section{OVERLAY}
Overlay berfungsi sebagai antarmuka yang ditampilkan pada platform streaming melalui OBS. Seluruh elemen overlay diambil secara \textit{real-time} dari API sehingga kreator dapat menampilkan interaksi donasi secara langsung kepada penonton. Sub bab ini menampilkan implementasi setiap komponen overlay.

\subsection{Overlay Notifikasi Donasi}
Overlay menampilkan notifikasi donasi secara \textit{real-time} yang diambil melalui polling API tampilan ini dikonfigurasi agar kompatibel dengan OBS untuk keperluan streaming.

\begin{figure}[htbp]
    \centering
    \includegraphics[width=0.6\textwidth]{images/notifikasi_donasi.png}
    \caption{Notifikasi Donasi}
    \label{fig:notifikasi_donasi}
\end{figure}

\subsection{Overlay Media Share}
Overlay Media Share menampilkan video YouTube yang diputar berdasarkan permintaan donor. Pada tampilan ini, video akan muncul di area utama layar, sementara informasi donasi – seperti nama donor, nominal, dan pesan singkat – tampil di bagian bawah kiri.

\begin{figure}[htbp]
    \centering
    \includegraphics[width=0.6\textwidth]{images/media_share.png}
    \caption{Media Share}
    \label{fig:media_share}
\end{figure}

\subsection{Overlay QR link Donasi}
Overlay menyediakan kode QR statis yang mengarahkan penonton langsung ke halaman donasi kreator. QR ini memfasilitasi penonton streamer untuk berdonasi tanpa harus mengetik tautan secara manual.

\begin{figure}[htbp]
    \centering
    \includegraphics[width=0.4\textwidth]{images/qr_donasi.png}
    \caption{QR}
    \label{fig:qr_donasi}
\end{figure}

\subsection{Overlay Leaderboard}
Overlay ini menampilkan leaderboard daftar pendukung terbesar (\textit{top donors}) atau ranking total donasi yang telah dioptimasi dengan limit dan sorting. Tampilan ini digunakan untuk memberikan apresiasi kepada donatur selama streaming.

\begin{figure}[htbp]
    \centering
    \includegraphics[width=0.6\textwidth]{images/leaderboard_donatur.png}
    \caption{Leaderboard Donatur}
    \label{fig:leaderboard_donatur}
\end{figure}

\section{DASHBOARD ADMIN}
Bagian ini menampilkan antarmuka yang digunakan admin untuk mengelola aktivitas pada platform, mulai dari pemantauan data donasi hingga pengelolaan creator dan proses payout. Dashboard admin terdiri dari tiga menu utama: Dashboard, Creator, dan Payout.

\subsection{Halaman Dashboard}
Halaman dashboard menampilkan ringkasan statistik utama sistem, seperti jumlah creator terdaftar, jumlah pengajuan payout, serta jumlah payout yang telah diselesaikan. Selain itu, halaman ini juga memvisualisasikan Top 5 Creator Paling Aktif berdasarkan total donasi yang diterima.

\begin{figure}[htbp]
    \centering
    \includegraphics[width=0.8\textwidth]{images/dashboard_admin.png}
    \caption{Dashboard Admin}
    \label{fig:dashboard_admin}
\end{figure}

\subsection{Halaman Creator}
Halaman creator digunakan untuk melihat daftar creator yang terdaftar pada platform, termasuk informasi status kelengkapan data payout masing-masing. Admin dapat melakukan pencarian creator dan melihat detail lengkap akun creator melalui dialog Detail Creator.

\begin{figure}[htbp]
    \centering
    \includegraphics[width=0.8\textwidth]{images/tabel_creator.png}
    \caption{Tabel Daftar Creator dan Detail Creator}
    \label{fig:tabel_creator}
\end{figure}

\subsection{Halaman Payout}
Halaman payout menampilkan daftar pengajuan pencairan dana yang dilakukan oleh para creator. Admin dapat memantau nominal, waktu pengajuan, status proses, serta catatan yang terkait dengan setiap payout. Halaman ini membantu admin memastikan bahwa seluruh proses pencairan berjalan dengan transparan dan terdokumentasi.

\begin{figure}[htbp]
    \centering
    \includegraphics[width=0.8\textwidth]{images/tampilan_payout.png}
    \caption{Tampilan Payout}
    \label{fig:tampilan_payout}
\end{figure}

\section{STRUKTUR BASIS DATA}
Sistem menggunakan MongoDB sebagai basis data utama. Setiap fitur ini menghasilkan koleksi tersendiri untuk memudahkan proses penyimpanan, pemantauan, dan analisis data. Koleksi yang terbentuk dan fungsinya dapat dilihat pada Tabel \ref{tab:struktur_database}.

\begin{table}[htbp]
    \centering
    \caption{Struktur Basis Data Nyumbangin}
    \label{tab:struktur_database}
    \begin{tabularx}{\textwidth}{|l|X|}
        \hline
        \textbf{Koleksi} & \textbf{Deskripsi Fungsi} \\ \hline
        admins & Menyimpan data akun admin untuk panel pengelolaan. \\ \hline
        contacts & Menyimpan feedback atau saran/keluhan dari pengguna. \\ \hline
        creators & Menyimpan informasi kreator, data payout, dan profil. \\ \hline
        donations & Mencatat transaksi sementara sebelum divalidasi. \\ \hline
        donationhistories & Menyimpan riwayat donasi yang sudah diproses (PAID). \\ \hline
        donationshares & Mencatat data pembagian link donasi. \\ \hline
        filteredwords & Menyimpan daftar kata-kata terlarang (sensor). \\ \hline
        mediashares & Menyimpan request video YouTube untuk diputar. \\ \hline
        monthlyleaderboards & Menyimpan agregasi peringkat donatur bulanan. \\ \hline
        notifications & Menyimpan data notifikasi real-time untuk overlay. \\ \hline
        payouts & Mencatat pengajuan dan proses pencairan dana. \\ \hline
        profileimages & Menyimpan metadata gambar profil kreator. \\ \hline
    \end{tabularx}
\end{table}

\include{chapters/conclusion}

% --- BIBLIOGRAPHY ---
\printbibliography[heading=bibintoc, title={DAFTAR PUSTAKA}]

\end{document}
